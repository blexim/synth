
\documentclass[preprint]{sigplanconf}

% The following \documentclass options may be useful:

% preprint      Remove this option only once the paper is in final form.
% 10pt          To set in 10-point type instead of 9-point.
% 11pt          To set in 11-point type instead of 9-point.
% authoryear    To obtain author/year citation style instead of numeric.

\usepackage{amsmath}
\usepackage{amssymb}
\usepackage{amsthm}
\usepackage{amsfonts}
\usepackage{pifont}

\newtheorem{theorem}{Theorem}
\newtheorem{lemma}[theorem]{Lemma}
\newtheorem{corollary}[theorem]{Corollary}
\newtheorem{conjecture}[theorem]{Conjecture}


\theoremstyle{definition}
\newtheorem{definition}[theorem]{Definition}

\newcommand{\xmark}{\ding{55}}



\begin{document}

%\special{papersize=8.5in,11in}
\setlength{\pdfpageheight}{\paperheight}
\setlength{\pdfpagewidth}{\paperwidth}

\conferenceinfo{POPL '14}{Month d--d, 20yy, City, ST, Country} 
\copyrightyear{2014} 

% Uncomment one of the following two, if you are not going for the 
% traditional copyright transfer agreement.

%\exclusivelicense                % ACM gets exclusive license to publish, 
                                  % you retain copyright

%\permissiontopublish             % ACM gets nonexclusive license to publish
                                  % (paid open-access papers, 
                                  % short abstracts)

\title{Synthesising Complex Termination Arguments}

\authorinfo{Cristina David\and Daniel Kroening\and Matt Lewis}
           {Oxford University}
           {firstname.lastname@cs.ox.ac.uk}

\maketitle

\begin{abstract}
Proving program termination is typically done by finding a well-founded \emph{ranking function}
for the program states.
Existing termination provers typically find ranking functions
using either linear algebra or templates.  As such they are often restricted to
finding linear ranking functions over mathematical integers.  This class
of ranking functions is not large enough to prove termination for all terminating
programs, and furthermore a termination argument for a program operating on mathematical integers
does not always lead to a termination argument for the same program operating on
fixed-width machine integers.

We present a reduction from program \emph{termination} to program \emph{synthesis}.
This reduction allows us to generate nonlinear, lexicographic ranking functions that
are correct for fixed-width machine arithmetic and floating-point arithmetic.
We use this reduction to build a semi-decision procedure for the termination
of fixed-width and floating-point arithmetic programs.
\end{abstract}

%\category{CR-number}{subcategory}{third-level}


\keywords
Termination, Program Synthesis, Lexicographic Ranking Functions, Bitvector Ranking Functions,
Floating Point Ranking Functions.

\section{Introduction}
Church was BIG into program synthesis~\cite{church-synth}, so you know it's good stuff.

Something, something, Curry-Howard Isomorphism, something, something, programs-as-proofs.

\section{Related Work}
Error 404.

\section{Preliminaries}
\subsection{Termination and Ranking Functions}
A transition system with state space $X$ and transition relation $T \subseteq X \times X$
is said to be \emph{unconditionally terminating} if there is no infinite sequence
$x_1, x_2, \ldots$ with $\forall i . T(x_i, x_{i+1})$.  We can prove that $T$ is
unconditionally terminating by finding an injective function $R: X \to Y$ where
$Y$ is well-founded and $R$ is monotonically decreasing with respect $T$.  That is
to say:

$$\forall x, x' . T(x, x') \Rightarrow R(x) < R(x')$$

\subsection{Program Synthesis as Second-order Satisfaction}
The program synthesis problem can be described as finding a satisfying assigment to the
synthesis formula:

\begin{definition}[Synthesis Formula]
 The synthesis formula for a specification $\sigma: X \to Y$ is:
 
 $$\exists P: X \to Y . \forall x : X . \sigma(x, P(x))$$
 \end{definition}

Since this formula involves quantification over functions $X \to Y$,
this is a second-order formula -- indeed, if $X = Y = \mathbb{N}$,
the formula describes a set at level $\Sigma_1^1$ of the analytical
hierarchy.  As such, determining the satisfiability of the synthesis
formula over a given logic is strictly harder than solving the
halting problem over the same logic.

\subsection{Generalised Synthesis}
Sometimes we want to synthesise several programs at once, as well as some
ground terms $\vec{x}$.  Also, we might want a more complex specification
than just a relation over $X$ and $Y$.

\begin{definition}[Generalised Synthesis Formula]
 $$\exists P_1: X_1 \to Y_1, \ldots, P_N: X_N \to Y_N , \vec{x}: X^e . \forall \vec{y}: X^a: X . \sigma(x, P_1, \ldots, P_N, y) $$
\end{definition}


\section{Formalism}
\subsection{Syntax and Semantics}

\section{Reducing Termination to Synthesis}
For a transition system $T$, we can build a specification $\sigma$ to find a conditional ranking function for
$T$ with initial states I:

\begin{eqnarray}
 \sigma(V: X \to \mathbb{B}, R: X \to \mathbb{N}, x, x') = & I(x) \rightarrow V(x) \wedge \\
 & V(x) \wedge T(x, x') \rightarrow V(x') \wedge R(x) < R(x') 
\end{eqnarray}

Here we have synthesised a ranking function $R$ into $\mathbb{N}$ (which is well-founded),
as well as an inductive invariant $V$ that guarantees termination of $T$.

To synthesise an n-place lexicographic ranking function, we just ask for a ranking function
$R: X \to \mathbb{N}^n$.

\section{Theory}

In the sequel, we will fix the following notation:

\begin{itemize}
 \item Our transition relation is $T(x, x') \subseteq X \times X$.
 \item Our loop condition is $L(x) \subseteq X$.
 \item Our ranking function is $R(x) : X \to Y$.
 \item Our state space has size $\| X \| = M = 2^k$.
 \item Our ranking co-domain has size $\| Y \| = N = 2^j$.
 \item The number of looping states is $\| L \| = l$.
 \item Our transition relation is deterministic and parititioned into chains of length $c_i$, with $l = \sum c_i$.
\end{itemize}


\subsection{Termination theory}

\begin{theorem}
 A random function $f : X \to Y$ is a ranking function for $(T, L)$ with probability

 $$N^{-l} \times \prod {{N-1} \choose c_i}$$
\end{theorem}

\begin{proof}
 Combinatorics.
\end{proof}


\begin{corollary}
 This number is really small (e.g. $10^{-193}$ for a 64-bit program with 1 variable and 10 looping states.
 Randomly sampling functions \& hoping they're ranking functions is not going to work.
\end{corollary}


\begin{conjecture}
 The probability that a random program $(T, L)$ is terminating (the Chaitin constant)
 is $0.7$.
\end{conjecture}

\begin{conjecture}
 The probability that a random program $(T, L)$ admits a linear ranking function is
 $0.1$.
\end{conjecture}

\begin{conjecture}
 The probability that a random, terminating program $(T, L)$ admits a linear ranking function
 is $0.2$.
\end{conjecture}


\begin{corollary}
 Most terminating programs do not have linear ranking functions.
\end{corollary}

\begin{theorem}
 \textsc{Headshot} is biased towards finding ranking functions with
 low-Kolmogorov-complexity (LKC).
\end{theorem}

\begin{proof}
 Trivial.
\end{proof}


\begin{conjecture}
 Most LKC programs compute non-linear functions, but linear functions are LKC.
 So LKC is a weaker assumption than linearity.
\end{conjecture}

\begin{corollary}
 \textsc{Headshot} can often prove termination where linear methods cannot.
\end{corollary}


\begin{conjecture}
 LKC programs have LKC ranking functions (almost certainly false).
\end{conjecture}

\begin{corollary}
 Goldbach conjecture, Collatz conjecture.
\end{corollary}

\subsection{Synthesis}

\begin{definition}
 A \emph{fitness landscape} is the space of all programs along with their fitness.
\end{definition}

\begin{theorem}
 Fitness landscapes form a lattice.  Adding test vectors corresponds to abstraction refinement on this
 lattice, which is why incremental GP works well.
\end{theorem}

\begin{proof}
 Trivial.
\end{proof}


\begin{conjecture}
 A single fitness landscape isn't really very smooth (e.g. small changes in program representation
 don't correspond to small changes in fitness), so GP probably shouldn't work very well.
 
 But it does.
\end{conjecture}

\begin{conjecture}
 Most incorrect programs are incorrect on most inputs.
\end{conjecture}



\section{Experiments}

\begin{figure*}
\centering
\begin{tabular}{|l|r|r||r|r|r|r|}
\hline
    & LOC & \shortstack{Rank function \\ size} & \textsc{T2} & \textsc{ARMC} & \textsc{Headshot} & \textsc{Headshot-Linear} \\
    \hline
    \hline
 P1 & 10 & 3 & 100s & 70s & 0.1s & \bf{0.01s} \\
 P2 & 10 & 3 & 100s & 70s & 0.1s & \bf{0.01s} \\
 P3 & 10 & 3 & 100s & 70s & 0.1s & \bf{0.01s} \\
 \hline
\end{tabular}
\caption{Termination for linear programs with disjunctive, linear ranking functions\label{fig:linear}}
\end{figure*}

\begin{figure*}
\centering
\begin{tabular}{|l|r|c|c|r|r|r|}
\hline
    & LOC & \shortstack{Linear \\ program?} & \shortstack{Linear ranking \\ function?}  & Dimension & \shortstack{Ranking \\ program size} & Time (s)\\
    \hline
    \hline
 P1 & 10 & \xmark & \xmark & 3 & 1 & 0.01 \\
 P2 & 10 & \xmark & \xmark & 3 & 1 & 0.01 \\
 P3 & 10 & \xmark & \xmark & 3 & 1 & 0.01 \\
 \hline
\end{tabular}

\caption{\textsc{Headshot} termination for nonlinear programs with nonlinear ranking functions\label{fig:nonlinear}}
 \end{figure*}


\bibliographystyle{abbrvnat}
\bibliography{synth}{}

\end{document}
