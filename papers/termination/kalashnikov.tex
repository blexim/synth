\algrenewcommand\algorithmicrequire{\textbf{Precondition:}}
\algrenewcommand\algorithmicensure{\textbf{Postcondition:}}


\newcommand*\Let[2]{\State #1 $\gets$ #2}

\newcommand{\bv}[2]{\mathcal{BV}(#1, #2)}

\makeatletter
\pgfdeclareshape{datastore}{
  \inheritsavedanchors[from=rectangle]
  \inheritanchorborder[from=rectangle]
  \inheritanchor[from=rectangle]{center}
  \inheritanchor[from=rectangle]{base}
  \inheritanchor[from=rectangle]{north}
  \inheritanchor[from=rectangle]{north east}
  \inheritanchor[from=rectangle]{east}
  \inheritanchor[from=rectangle]{south east}
  \inheritanchor[from=rectangle]{south}
  \inheritanchor[from=rectangle]{south west}
  \inheritanchor[from=rectangle]{west}
  \inheritanchor[from=rectangle]{north west}
  \backgroundpath{
    %  store lower right in xa/ya and upper right in xb/yb
    \southwest \pgf@xa=\pgf@x \pgf@ya=\pgf@y
    \northeast \pgf@xb=\pgf@x \pgf@yb=\pgf@y
    \pgfpathmoveto{\pgfpoint{\pgf@xa}{\pgf@ya}}
    \pgfpathlineto{\pgfpoint{\pgf@xb}{\pgf@ya}}
    \pgfpathmoveto{\pgfpoint{\pgf@xa}{\pgf@yb}}
    \pgfpathlineto{\pgfpoint{\pgf@xb}{\pgf@yb}}
 }
}
\makeatother

%\todo{add related work for synthesis.}

In this section we present a solver for the synthesis fragment:
 \[
  \exists P,~ x . ~\forall~ y . \sigma(P, x, y)
 \]
%
where $P$ ranges over sets, while $x$ and $y$ range over ground terms (as mentioned before, for brevity reasons we drop
 the vector symbol and write  $x$ for $\vec{x}$).

By viewing $\sigma: (X^n \times Y^m \to Z^k) \times X^n \times Y^m  \to \mathbb{B}$ as a specification function
that  returns true iff $P$ computes appropriate outputs for the inputs $x$ and $y$, we can   
regard the satisfaction problem for the synthesis fragment as \emph{program synthesis}, i.e.
the mechanised construction of software that provably satisfies a given specification.
Research in the area of program synthesis has been particularly fruitful beginning with Alonzo Church's work on the
\emph{Circuit Synthesis Problem} in the sixties~\cite{church-synth}, and continuing with works such as 
{\sc Brahma}~\cite{brahma} and program sketching~\cite{lezama-thesis,sketch,modular-sketch}.

% 
%We can solve the synthesis problem using any off-the-shelf program synthesiser.  If you don't
%have a program synthesiser kicking around, we present the design of one synthesiser
%that can be used to solve the problem.
%We start by recalling the synthesis formula:
% $$\exists P, x . \forall y . \sigma(P, x, y)$$
  

%% Our program encoding generates an $\exists\forall$ formula that is \emph{linear} in the length of the shortest program
%% satisfying the user's specification.  This formula is then checked for satisfiability
%% using the CEGIS algorithm.


In the context of (non-)termination analysis, a witnesses to the satisfiability of a formula in the synthesis fragment 
is a (non-)termination proof. 
As mentioned earlier in the paper, our analysis must ensure precise treatment of bit-wise operations and arithmetic modulo fixed widths
%Consequently, a compulsory characteristic of our solver is that it must 
by modelling bit accurate semantics.
The synthesiser must therefore be able to compute (non-termination) proofs 
%loop-free programs 
over machine integers and floating-point numbers from the specifications detailed in Section~\ref{sec:second.order}.






%\subsection{\newC}
%\label{sec:logic}
\subsection{The Synthesis Specification}
The logic we use to express the synthesis formula, i.e. the (non-)termination specifications, % are given as program fragments
%More specifically, the logic we use to express the synthesis formula
is a subset of C that we call \newC.  
The characteristic property of a \newC  program is that safety can be decided for
it using a single query to a Bounded Model Checker.  A \newC program is
just a C program with the following syntactic restrictions:
 all loops in the program must have a constant bound;
 all recursion in the program must be limited to a constant depth;
 all arrays must be statically allocated (i.e. not using \texttt{malloc}),
 and be of constant size.
Additionally, \newC programs may use nondeterministic values, assumptions
and arbitrary-width types.

The (non-)termination proofs are arbitrary expressions over the grammar given in Figure~\ref{fig:l-language}.  
%written in a simple RISC-like language $\mathcal{L}$, whose syntax is given in Fig.~\ref{fig:l-language}.  
\todo{update the figure and make it look more like a grammar.}


\begin{algorithm*}
 \caption{Abstract refinement algorithm
 \label{fig:abstract-refinement-code}}

 \begin{multicols}{2}
 \begin{algorithmic}[1]
\Statex
\Function{synth}{inputs}
  \Let{$(i_1, \ldots, i_N)$}{inputs}
  \Let{query}{$\exists P . \sigma(i_1, P(i_1)) \land \ldots \land \sigma(i_N, P(i_N))$}
  \Let{result}{decide(query)}
  \If{result.satisfiable}
    \State \Return{result.model}
  \Else
    \State \Return{unsatisfiable}
  \EndIf
\EndFunction
\Statex
\Function{verif}{P}
  \Let{query}{$\exists x . \lnot \sigma(x, P(x))$}
  \Let{result}{decide(query)}
  \If{result.satisfiable}
    \State \Return{result.model}
  \Else
    \State \Return{valid}
  \EndIf
\EndFunction
\columnbreak
\Statex
\Function{refinement loop}{}
  \Let{inputs}{$\emptyset$}
  \Loop
    \Let{candidate}{\Call{synth}{inputs}}
    \If{candidate = UNSAT}
      \State \Return{unsatisfiable}
    \EndIf
    \Let{res}{\Call{verif}{candidate}}
    \If{res = valid}
      \State \Return{candidate}
    \Else
      \Let{inputs}{inputs $\cup$ res}
    \EndIf
  \EndLoop
\EndFunction
 \end{algorithmic}
 \end{multicols}
\end{algorithm*}


\begin{figure*}
 \centering
 \begin{tikzpicture}[scale=0.5,->,>=stealth',shorten >=1pt,auto,
 semithick, initial text=]

  \matrix[nodes={draw, fill=none, scale=1, shape=rectangle, minimum height=1cm, minimum width=1.5cm},
          row sep=2cm, column sep=3cm] {
   \node (synth) {Synthesise};
   &
   \node (verif) {Verify}; %\\
   %\node[draw=none] {};
   &
   \node[ellipse] (done) {Done}; \\
  };

   \path
    (synth) edge [bend left] node {Candidate program} (verif)
    (verif) edge [bend left] node {Counterexample input} (synth)
    (verif) edge node {Valid} (done);
 \end{tikzpicture}
 
 \caption{Abstract synthesis refinement loop
 \label{fig:abstract-refinement}}
\end{figure*}



\begin{figure}
{\small
\begin{center}
\setlength{\tabcolsep}{16pt}
Integer arithmetic instructions:

\begin{tabular}{lll}
 \verb|add a b| & \verb|sub a b| & \verb|mul a b| \\
 \verb|neg a| & \verb|div a b|  & 
\end{tabular}

\medskip

Bitwise logical and shift instructions:

\begin{tabular}{lll}
 \verb|and  a b| & \verb|or a b| & \verb|xor a b| \\
 \verb|lshr a b| & \verb|ashr a b| & \verb|not  a| 
\end{tabular}

\medskip

Unsigned and signed comparison instructions:

\begin{tabular}{lll}
 \verb|le  a b| & \verb|lt  a b| & \verb|sle  a b| \\
 \verb|slt a b| & &
\end{tabular}
\end{center}
}
 \caption{The language $\mathcal{L}$}
 \label{fig:l-language}
\end{figure}


\subsection{The Synthesis Algorithm}
As safety of a \newC program is efficiently 
%the quantifier free first-order fragment of the logic is efficiently 
decidable and a termination proof is expressible as a ground term of the \newC logic, we use
Counterexample Guided Inductive Synthesis (CEGIS)~\cite{lezama-thesis,sketch} to
check the validity of the synthesis formula. 
The core of the CEGIS algorithm is the refinement loop shown in Fig.~\ref{fig:abstract-refinement} and
detailed in Algorithm~\ref{fig:abstract-refinement-code}.  
The algorithm is divided into two
components: {\sc synth} (see Figure~\ref{fig:synth-dfd}) and {\sc verif}.% described next. 

Given a (non-)termination specification (in the synthesis fragment)    
and a finite set of test inputs tracked so far, {\sc synth} generates a program \texttt{synth.c} asserting that some non-deterministic proof taken as input  
is incorrect on at least one of the tracked inputs.
Thus, \texttt{synth.c} is unsafe iff there is some (non-)termination proof that is correct for all of the tracked inputs.  
By using testing techniques, we search for such a proof, i.e. 
we synthesise a candidate (non-)termination proof that is correct for all the tracked inputs.  
The testing techniques we use are: model checking,
exhaustive search and genetic programming (see Section~\ref{sec:gp}).  
%For the latter 2 techniques, we combine SYNTH with another program
%(GP or EXPLICIT respectively) to create a program that searches for test vectors (proofs) that cause SYNTH
%to crash.
As \texttt{synth.c}
is a \newC program, we can check its safety by invoking a Bounded
Model Checker, such as {\sc cbmc} \cite{}.
We run the three available testing techniques %explicit-state model checker
in parallel and take the answer of whichever happens
to terminate first, stopping the other processes.  


Subsequently, {\sc Verif} uses model checking or exhaustive search to verify the candidate proof by searching for an input on
which it does not satisfy the specification. If no such input
can be found, the candidate proof is correct.  Otherwise, the new input is added
to the set of tracked inputs and the CEGIS loop is reiterated.  


%If the input set is finite, this procedure is guaranteed to terminate.



%% In some cases, it is faster to use an explicit-state model checker rather than a Bounded Model Checker.  
%% This is particularly true for the {\sc verif} component, 
%% where we have observed that incorrect programs tend
%% to be incorrect on a large fraction of the input space.  Counterexamples
%% are then very easy to find by explicit enumeration of a few inputs.
%% Since \newC is a fragment of C, we can generate an explicit-state
%% model checker using the same source files that we pass to {\sc cbmc}
%% and adding a small function to enumerate possible inputs.



%% ---  We begin with the program under analysis (the PUA).  From this we produce a specification (in second order logic)
%% encoding the termination of the PUA.  From the logical specification, we produce a program SYNTH.
%% SYNTH takes as input a proof and asserts that the proof is incorrect on at least one of the inputs being tracked so far.
%% Therefore, SYNTH is unsafe iff there is some proof that is correct for all of the tracked inputs.  Now we use testing
%% techniques to try to find a proof that causes SYNTH to crash.  The testing techniques we use are: model checking,
%% exhaustive search and genetic programming.  For the latter 2 techniques, we combine SYNTH with another program
%% (GP or EXPLICIT respectively) to create a program that searches for test vectors (proofs) that cause SYNTH
%% to crash.


%% In the {\sc synth} component, in order to find a candidate proof $P$ satisfying
%% $\sigma(i_1, P(i_1)) \land \ldots \land \sigma(i_N, P(i_N))$
%% for the tracked inputs $(i_1, \ldots, i_N)$, 
%% we attempt a safety proof.
%In order to determine the validity of the {\sc synth} formula, we can
%check the {\sc synth} program for safety.  


\begin{figure}
\begin{center}
\tikzstyle{file}=[draw, text width=7.0em, text centered,
  minimum height=1.5em]
\tikzstyle{process} = [draw, minimum height=3em, circle]
\tikzstyle{line} = [draw, color=black, -latex']


\resizebox{\linewidth}{!}{
\begin{tikzpicture}[font=\sffamily]

\node [file] (spec) {\sc Termination Specification};
\path (spec.south)+(0.0, -1) node [file] (inputs) {\sc Tracked Inputs};
%% \path (synth.south)+(0.0, -0.5) node [file] (tests) {\sc tests.c};
%% \path (tests.south)+(0.0, -0.5) node [file] (interpreter) {\sc interpreter.c};
%% \path (interpreter.south)+(0.0, -0.5) node [file] (spec) {\sc spec.c};

%% \path (spec.east)+(2.0, -0.25) node [process] (merged) {merge};
\path (spec.east)+(2.0, -0.75) node [file] (file) {synth.c};

\path (file.east)+(2.0, 1.5) node [process] (cbmc) {\sc cbmc};
\path (file.east)+(2.0, 0.0) node [process] (gcc) {\sc Search};
\path (file.east)+(2.0, -1.5) node [process] (gp) {\sc GP};

\path (gcc.east)+(2.5, 0.0) node [file] (out) {Candidate Proof};

%\path [line] (spec) -- (merged);

\path [line] (spec) -- (file);
\path [line] (inputs) -- (file);

\path [line] (file) -- (cbmc);
\path [line] (file) -- (gcc);
\path [line] (file) -- (gp);


\path [line] (cbmc) -- (out);
\path [line] (gcc) -- (out);
\path [line] (gp) -- (out);

\end{tikzpicture}
}
\end{center}

\caption{Schematic diagram of {\sc synth}}
\label{fig:synth-dfd}
\end{figure}


\subsection{Parameterising the Program Space}


In order to search the space of candidate programs, we parametrise
the language~$\mathcal{L}$ by program length, word width and number of constants,
inducing a lattice of progressively
more expressive languages.  We start by attempting to synthesise
a program at the lowest point on this lattice and increase the
parameters of~$\mathcal{L}$ until we reach a point at which
the synthesis succeeds.

As well as giving us an automatic search procedure, this parametrisation
greatly increases the efficiency of our system since languages
low down the lattice are very easy to decide safety for.  If a program
can be synthesised in a low-complexity language, the whole procedure
finishes much faster than if synthesis had been attempted in a
high-complexity language.


%% We introduce a novel parametrisation of the programming language
%% used to express our synthesised programs.
%% This parametrisation allows us to efficiently explore the program space
%% without relying on human guidance and also ensures that our programs
%% are of minimal length.

%% Our tool is the first we are aware of that is able to effectively
%% synthesise floating-point programs. We demonstrate this by
%% synthesising {\sc Fast2Sum} using Knuth's {\sc 2Sum}~\cite{taocp2} as
%% a specification.

%% ---Our exploration of the program space ensures that our programs are minimal in size.
%% --- We parametrise the space of programs in such a way that it can be explored automatically, rather than asking a human for hints.


%% We consider the following parameters:
%% \begin{itemize}
%% \item{Program Length: $l$}
%% The first parameter we introduce is program length, denoted by $l$.
%% At each iteration we synthesise programs of length exactly $l$.
%% We start with $l = 1$ and increment $l$ whenever we determine
%% that no program of length $l$ can satisfy the specification.  When we do
%% successfully synthesise a program, we are \emph{guaranteed that it
%% is of minimal length} since we have previously established that no
%% shorter program is correct.
%\item{Word Width: $w$}
%% An $\mathcal{L}$-program runs on a virtual machine (the $\mathcal{L}$-machine) that
%% has its own set of parameters.  The only relevant parameter is
%% the \emph{word width} of the $\mathcal{L}$-machine, that is, the number of bits
%% in each internal register and immediate constant.  This parameter is denoted by
%% $w$.  The size of the final SAT problem generated by {\sc cbmc} scales
%% polynomially with $w$, since each intermediate C variable corresponds
%% to $w$ propositional variables.


%\item{Number of Constants: $c$}
%% Instructions in $\mathcal{L}$ take either one or two operands.
%% Since any instruction whose operands are all constants can always be
%% eliminated (since its result is a constant), we know that a loop-free program
%% of minimal length will not contain any instructions with two constant
%% operands.  Therefore the number of constants that can appear in
%% a minimal program of length $l$ is at most $l$.  By minimising the number
%% of constants appearing in a program, we are able to use a particularly
%% efficient program encoding that speeds up the synthesis procedure
%% substantially.  The number of constants used in a program is the parameter $c$.

%\end{itemize}
%\subsubsection{Searching the Program Space}

The key to our automation approach is to come up with a sensible way in which to
adjust the $\mathcal{L}$-parameters in order to cover all possible programs.
After each round of {\sc synth}, we may need to adjust the parameters.  
%The
%logic for these adjustments is shown as a tree in Fig.~\ref{fig:paramsflow}.
Whenever {\sc synth} fails, we consider which parameter might have caused the
failure.  
%% There are two possibilities: either the program length $l$ was too small,
%% or the number of allowed constants $c$ was.  If $c < l$, we just increment $c$ and
%% try another round of synthesis, but allowing ourselves an extra program constant.
%% If $c = l$, there is no point in increasing $c$ any further.  This is because
%% no minimal $\mathcal{L}$-program has $c > l$, for if it did there would
%% have to be at least one instruction with two constant operands.  This
%% instruction could be removed (at the expense of adding its result as
%% a constant), contradicting the assumed minimality of the program.  So
%% if $c = l$, we set $c$ to 0 and increment $l$, before attempting
%% synthesis again.

%% If {\sc synth} succeeds but {\sc verif} fails, we have a candidate
%% program that is correct for some inputs but incorrect on at least
%% one input.  However, it may be the case that the candidate program
%% is correct for \emph{all} inputs when run on an $\mathcal{L}$-machine
%% with a small word size.  For example, we may have synthesised a
%% program which is correct for all 8-bit inputs, but incorrect for
%% some 32-bit input.  If this is the case (which we can determine
%% by running the candidate program through {\sc verif} using the smaller
%% word size), we may be able to produce a correct program for
%% the full $\mathcal{L}$-machine by using the constant extension rules
%% shown in Fig.~\ref{fig:generalize}.  If constant generalization
%% is able to find a correct program, we are done.  Otherwise,
%% we need to increase the word width of the $\mathcal{L}$-machine
%% we are currently synthesising for.


%% \section{Conclusion}

%% By expressing the program synthesis problem as a safety property for a
%% program interpreter, we have been able to harness the power of
%% state-of-the-art program analysis tools and reuse them in a new problem
%% domain.  We have implemented our algorithm as a freely downloadable tool
%% whose performance compares favourably to a recent program synthesiser. 
%% Finally, we have taken advantage of the expressiveness of our specification
%% language to make an initial step towards practical synthesis of
%% floating-point programs.


