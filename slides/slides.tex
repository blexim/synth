% Copyright 2009 by Tomasz Mazur
%
% This file may be distributed and/or modified in all ways.


\documentclass[xcolor=pdftex,t,11pt]{beamer}

%%%%%%%%%%%%%%%%%%%%%%%%%%%%%%%%%%
%       SET OPTIONS BELOW        %
%%%%%%%%%%%%%%%%%%%%%%%%%%%%%%%%%%

\usetheme[
% Toggle showing page counter
pagecounter=true,
%
% String to be used between the current page and the
% total page count, e.g. of, /, from, etc.
pageofpages=of,
%
% Defines the shape of bullet points. Available options: circle, square
bullet=circle,
%
% Show a line below the frame title. 
titleline=true,
%
% Set the style of the title page (true for fancy, false for standard)
alternativetitlepage=true,
%
% Institution logo for fancy title page.
% Comment out to remove the logo from the title page.
% IMPORTANT: THERE IS A BUG IN SOME VERSIONS OF PDFLATEX AND FONTS
% ON THE LOGOS ARE NOT RENDERED PROPERLY. IN SUCH A CASE ADD `2` 
% TO THE NAME OF THE LOGO, E.G. comlab2 INSTEAD OF comlab
%titlepagelogo=images/titlepage/ou,
%
% Department footer logo for fancy title page
% Comment out to remove the logo from the footer of the title page/
% IMPORTANT: THERE IS A BUG IN SOME VERSIONS OF PDFLATEX AND FONTS
% ON THE LOGOS ARE NOT RENDERED PROPERLY. IN SUCH A CASE ADD `2` 
% TO THE NAME OF THE LOGO, E.G. comlab2 INSTEAD OF comlab
%titlepagefooterlogo=images/titlepage/comlab,
%
% Institution/department logo for ordinary slides
% Comment this line out to remove the logo from all the pages.
% Available logos are: ou, comlab, comlabinline, comlabou
% IMPORTANT: THERE IS A BUG IN SOME VERSIONS OF PDFLATEX AND FONTS
% ON THE LOGOS ARE NOT RENDERED PROPERLY. IN SUCH A CASE ADD `2` 
% TO THE NAME OF THE LOGO, E.G. comlab2 INSTEAD OF comlab
%ordinarypageslogo=TU-Signet,
%
%
% Add watermark in the bottom right corner
%watermark=<filename>,
%
% Set the height of the watermark.
%watermarkheight=100pt,
%
% The watermark image is 4 times bigger than watermarkheight.
%watermarkheightmult=4,
]{Torino}

% Select color theme. Available options are:
% mininmal, greenandblue, blue, red
\usecolortheme{blue}

%Select different font themes.Available options are:
% default, serif, structurebold, structureitalicserif, structuresmallcapsserif
\usefonttheme{structurebold}

\usepackage{tikz}
\usepackage{pgf}

\pgfdeclareimage[height=6ex]{ou-logo}{images/ou}

\logo{\pgfuseimage{ou-logo}}

\usepackage{listings}

\lstset{language=C,
 basicstyle=\tiny
 }

\usepackage{ifthen}

\usepackage{moreverb}
\usepackage{pgf}
\usepackage{tikz}
\usetikzlibrary{arrows, automata, shapes.multipart, chains, positioning, fit}

\usepackage{color}
\definecolor{light-gray}{gray}{0.80}

%%%%%%%%%%%%%%%%%%%%%%%%%%%%%%%%%%
%       PRESENTATION INFO        %
%%%%%%%%%%%%%%%%%%%%%%%%%%%%%%%%%%

\author{Daniel Kroening \and Matt Lewis}
\title{Program Synthesis with Bounded Model Checking}
%\subtitle{}
\institute{Oxford University}
\date{\today}

\begin{document}



%%%%%%%%%%%%%%%%%%%%%%%%%%%%%%%%%%
%       SLIDE DEFINITIONS        %
%%%%%%%%%%%%%%%%%%%%%%%%%%%%%%%%%%

\begin{frame}[plain]
  \titlepage
\end{frame}

\begin{frame}{Outline}
  \tableofcontents
\end{frame}

\section{Motivation}


\begin{frame}[fragile]{Motivation}
What does this program do?

\begin{center}
\begin{minipage}{0.3\linewidth}
 \begin{lstlisting}[language=C,basicstyle=\normalsize]
int f(int x) {
  return -x & x;
}
 \end{lstlisting}
\end{minipage}
\end{center}

\pause

Correct!  It isolates the \emph{least signficant bit} of \verb|x| that is set:


\begin{center}
\begin{tabular}{llcccccccc}
 x       & = & 1 & 0 & 1 & 1 & 1 & 0 & 1 & 0 \\
 -x      & = & 0 & 1 & 0 & 0 & 0 & 1 & 1 & 0 \\
 x \& -x & = & 0 & 0 & 0 & 0 & 0 & 0 & 1 & 0
\end{tabular}
\end{center}


\end{frame}

\begin{frame}[fragile]{Motivation}
How about this one?

\begin{center}
\begin{minipage}{0.75\linewidth}
 \begin{lstlisting}[language=C,basicstyle=\normalsize]
int g(int x, y) {
  return x - ((x - y) * (x <= y));
}
 \end{lstlisting}
\end{minipage}
\end{center}

\pause

Correct again!  It computes the \emph{max} of \verb|x| and \verb|y|.
\end{frame}

\begin{frame}{Motivation}
 Finding programs like that is hard.  Wouldn't it be nice if we could do it automatically?
\end{frame}


\section{Abstract Program Synthesis}
\begin{frame}{The Synthesis Formula}
Our high-level goal is to synthesise programs that meet some specification.

\pause

Abstractly, a program $P$ computes some function from an input set $I$ to an output set $O$:
$$P : I \rightarrow O$$

\pause

A specification is a relation between inputs and outputs:
$$\sigma \subseteq I \times O$$

\pause

The synthesis problem is to find a program that meets the specification on all inputs:
$$\exists P . \forall x \in I . \sigma(x, P(x))$$
\end{frame}

\begin{frame}[fragile]{Solving the Synthesis Formula with CEGIS}
 Solving second order formulae is hard.  Instead we'll alternately solve two \emph{first-order} formulae
 in a refinement loop:

 \vspace{1cm}
 
  \begin{tikzpicture}[->,>=stealth',shorten >=1pt,auto,
 semithick, initial text=]

  \matrix[nodes={draw, fill=none, scale=1, shape=rectangle, minimum height=1cm, minimum width=1.5cm},
          row sep=2cm, column sep=3cm] {
   \node (synth) {Synthesise};
   &
   \node (verif) {Verify}; %\\
   %\node[draw=none] {};
   &
   \node[ellipse] (done) {Done}; \\
  };

   \path
    (synth) edge [bend left] node {Candidate program} (verif)
    (verif) edge [bend left] node {Counterexample input} (synth)
    (verif) edge node {Valid} (done);
 \end{tikzpicture}
\end{frame}

\begin{frame}{First-order Synthesis}
 If we have some small set of test inputs $\{ x_0, \dots, x_N\}$, we can find a program that is correct on just that set:

 $$\exists P . \sigma(x_0, P(x_0)) \wedge \dots \wedge \sigma(x_N, P(x_N))$$
\end{frame}

\begin{frame}{First-order Verification}
 If we have some program $P$ that might be correct, we can check whether it is in fact correct:
 
 $$\exists x . \lnot \sigma(x, P(x))$$
 
 If this formula is \emph{satisfiable}, $P$ is \emph{not} correct and we have found an input on which it is incorrect.
 We can add this input to our set of test inputs and loop back to synthesising a new program.
\end{frame}

\section{Concrete Synthesis in C}
\begin{frame}{Concrete Synthesis in C}
 We're using C as an implementation language.  To do that we need:
 
 \begin{itemize}
  \item A way of encoding programs as \emph{finite structures} in C.
  \item A way of encoding the first-order synthesis and verification formulae in C.
  \item A way of checking properties of C programs.
 \end{itemize}
\end{frame}

\begin{frame}[fragile]{Encoding Programs in C}
 To encode the programs we will synthesise, we just make a very simple language and an interpreter for it.  Programs
 in our language look like this:
 
 \vspace{0.7cm}

\begin{center}
\begin{minipage}{0.3\linewidth}
\begin{verbatimtab}
t1 = neg x
t2 = add t1 1
return t2
\end{verbatimtab}
\end{minipage}
\end{center}

\end{frame}

\begin{frame}{Encoding the First-Order Formulae in C}
 We encode the synthesis and verification formulae as \emph{safety properties} of C programs.
 
 We use CBMC as a decision procedure to check the safety of these programs.
\end{frame}

\section{Code Samples and Demos}

\begin{frame}{Demos!}

\end{frame}


\end{document}